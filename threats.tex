\section{Threats to Validity}\label{sec:threats}
\todo{In this section, the most prominent threats to validity of your study should be discussed.}
In this section, we discuss the most prominent threats to validity of this study.

\textbf{Internal validity.}
The internal validity threat is related to the design and execution of the literature review \cite{syslit3}. This threat is mitigated by defining a detailed research protocol in Section~\ref{sec:design} and locking this protocol so that it cannot change after the review is started to prevent personal bias.

\textbf{External validity.}
The external validity threat is related to the generalizability of the results \cite{syslit3}. More specifically, for a literature study this would mean that the selected studies not being representative of the full population (i.e., all the available published literature). To mitigate this form of bias, we used a generic search query (Figure~\ref{fig:search-string-refined}) inside of a search engine that indexes scientific literature from multiple sources. Additionally, we utilized both forward and backward snowballing \cite{syslit4} to further expand our selected literature with studies our automatic search might have missed.


\textbf{Construct validity.}
The construct validity threat concerns the relation between theory and observation. We mitigated this bias by searching multiple sources using Scopus with only general terms in our search string. Additionally, we rigorously selected relevant studies according to inclusion and exclusion criteria \cite{syslit1,syslit2,syslit3}.

\textbf{Conclusion validity.}
The conclusion validity concerns the degree in which our conclusions are reasonable based on our extracted data. We mitigated this by iteratively defining and updating our model during the data extraction and synthesis phase. Additionally, this threat was mitigated by applying well-known systematic literature review methods \cite{syslit1,syslit2, syslit3, syslit4}.